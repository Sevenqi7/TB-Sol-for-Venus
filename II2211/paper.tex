\documentclass[10pt,twocolumn]{article}

% Pacchetti per font e formattazione
\usepackage{times}        % Times New Roman
\usepackage[utf8]{inputenc}
\usepackage[T1]{fontenc}
\usepackage{geometry}
\geometry{a4paper, margin=2cm}
\usepackage{titlesec}
\usepackage{multicol}
\usepackage{cite}

% Titoli sezioni (formato)
\titleformat{\section}{\bfseries\fontsize{12}{14}\selectfont}{\thesection.}{0.5em}{}
\titleformat{\subsection}{\bfseries\fontsize{11}{13}\selectfont}{\thesubsection.}{0.5em}{}

% Dati frontespizio
\title{\textbf{Project Title}\\[0.5em]
\large Project subtitle if any}

\author{
Author #1 (auth #1 Email address), Author #2 (Email address), …
}

\date{}

\begin{document}

\maketitle

\section*{Abstract}
<Contains short summary of the paper>\\
Main body – Double column format\\
Text – Font – Times New Rom, 10 pts\\
Section Headers – Bold Font – Times New Roman, 12 pts\\
Subsection Headers – Bold font – Times new Roman, 11 pts\\
Paper length target 4-6 pages.

\section*{Keywords}
<optional section to help search engines to find the paper>

% Sezioni principali
\section{Introduction}
<Answers the question: Why do I do this research? The section may also contain which methodology have been used? Any research ethics? Sustainability issues?>

\section{Related Work}
<Answers the question: What have others done previously in do this research topc?>\\
<Literature study, refer to each reference [1] where it makes sense>
\subsection{riscv}
Researchers from TU graz developed FazyRV[1], a minimal-area open-source RV32I RISC-V core targeting IoT and low-workload applications, addressing the problem that 32-bit RISC-V processor cores reach a boundary on their minimal size. Their goal was to minimize area demand while fulfilling performance requirements and close the gap between prevalent 32-bit and 1-bit-serial RISC-V cores.\\
Qian Wei and his colleagues created a comprehensive survey of RISC-V instruction set architecture extensions because while RISC-V is popular for embedded processors [2], there is still a gap between RISC-V's capabilities and the requirements of various emerging computing scenarios like artificial intelligence and cloud computing.

\subsection{processors for venus}
Current research on high-temperature electronics for Venus has shown that SiC ICs can sustain operation for over a year at 500 °C and for months in simulated Venus conditions, supporting sensors and simple microprocessors. This demonstrates the feasibility of long-duration surface missions and motivates concepts like LLISSE targeting low-power seismic monitoring.[3] \\
At the processor level, studies of SiC-based computing infrastructures reveal that current prototypes achieve significantly lower throughput than space-proven silicon systems, yet provide guidelines at the microarchitecture and ISA levels for developing processors able to operate under Venus’s extreme environment.[4]\\
\subsection{FPGA}

\section{Specification}
<Outlines the constraints this work has, if any>\\

\section{Architecture}
<How do I/we plan to build this gadget given the constraints I have?. This section is sometimes split into subsections like Hardware and Software>

\section{Experiments}
<What experiments have I done? This is more or less a specification of the experiments.>

\section{Results}
<What results did I get from the experiments. This section is sometimes merged with the experiments section>

\section{Conclusion}
<This is a retrospective look at the introduction, and should present answers to the questions asked there and throughout the paper.>

\subsection{Future Work}
<Optional section that describe possible future work>

\section*{References}
[1] FazyRV: Closing the Gap between 32-Bit and Bit-Serial RISC-V Cores with a Scalable Implementation. / Kissich, Meinhard; Baunach, Marcel Carsten.
Proceedings of the 21st ACM International Conference on Computing Frontiers, CF 2024. Association for Computing Machinery (ACM), 2024. p. 240-248 (Proceedings of the 21st ACM International Conference on Computing Frontiers, CF 2024).
Research output: Chapter in Book/Report/Conference proceeding › Conference paper ›
[2] E. Cui, T. Li and Q. Wei, "RISC-V Instruction Set Architecture Extensions: A Survey," in IEEE Access, vol. 11, pp. 24696-24711, 2023, doi: 10.1109/ACCESS.2023.3246491.
keywords: {Instruction sets;Microprocessors;Computer architecture;Task analysis;Graphics processing units;Cloud computing;Artificial intelligence;RISC-V;instruction set architecture;extensions;survey},
[3] H. Kim, J. Bagherzadeh and R. G. Dreslinski, "SiC Processors for Extreme High- Temperature Venus Surface Exploration," 2022 Design, Automation & Test in Europe Conference & Exhibition (DATE), Antwerp, Belgium, 2022, pp. 406-411, doi: 10.23919/DATE54114.2022.9774769.
[4] Kremic, T. (2021). High Temperature Electronics for Venus Surface Applications: A Summary of Recent Technical Advances. https://doi.org/10.3847/25C2CFEB.E3883E19

\end{document}

